\documentclass[line,margin,UTF8]{res}
\usepackage{xcolor}
\usepackage{helvetica}  
\usepackage{enumitem}
\usepackage[hyperfootnotes=true,pdfborder={0 0 0}]{hyperref}  

\usepackage{xeCJK}
\setCJKmainfont[BoldFont=STSong, ItalicFont=STKaiti]{STSong}
\setCJKsansfont[BoldFont=STHeiti]{STXihei}
\setCJKmonofont{STFangsong}

\hypersetup{pdfborder=0 0 0}
\renewcommand\thefootnote{\textcolor{red}{\arabic{footnote}}}
 
\makeindex
 
 %AOP三剑客。javapoet,idleHandler

\begin{document}
\name{程思敏
(北京 男 1989年)
}
 
%\address{763585627@qq.com}
\address{15718887533}

\begin{resume}
%\section{求职意向}
%Android系统研发工程师
 

 %\section{自我介绍}
 %\begin{itemize}
 %\item 专注于android端的技术研发,在小米担经历过米聊和小米直播两款产品的核心开发,小米直播SDK技术负责人。带领直播客户端在代码架构上完成模块化、组件化、插件化的技术演变,在移动端长链接数据通道,消息系统,音视频业务,图片处理,性能优化等方面有丰富的经验和技术沉淀。目前在北京帧趣科技负责安卓研发。
 %\end{itemize}

\section{项目经历} 
{\bf 帧趣科技,客户端负责人,北京 \hfill  2019.02 $\sim$ 至今}
 \vspace{3pt}

  \begin{itemize}
 \item \makebox[7cm][l]{\textit{撕歌App\hfil}} 2019.01 $\sim$ 至今
 	\vspace{-3pt}
 	\begin{itemize}
 		\item
 		以技术合伙人的身份加入帧趣创业团队,负责skr撕歌app从无到有的架构设计,关键性技术验证,使用模块化组件化的设计很好地解耦了各项业务功能,为业务的快速迭代提供了很好的底层设施保障,在一年迭代52个版本的节奏下,代码质量,包大小,内存性能都得到了很好的维护。

		在组内营造良好的技术氛围,鼓励工具化效率化,
		%将内存分析,打包构建(120秒缩减到15秒),上传发布等落地成脚本,
		促使组内完成java->kotlin的开发语言切换,在ios工期紧张时,引入flutter混合开发减小两端的开发量。

 \end{itemize}
 \end{itemize}

{\bf 小米互联网四部,直播SDK技术负责人,北京 \hfill  2018.08 $\sim$ 2019.2}
 \vspace{3pt}

  \begin{itemize}
 \item \makebox[7cm][l]{\textit{直播助手SDK\hfil}} 2018.02 $\sim$ 至今
 	\vspace{-3pt}
 	\begin{itemize}
 		\item 
 		直播助手SDK是miui的内置组件,负责直播助手sdk的开发,为宿主提供快速搭建直播业务系统的能力,功能包括直播间推拉流,直播列表,房间弹幕礼物等,提供APK AAR方式供宿主接入,接入方有小米游戏,小米音乐。
 		其中APK接入方式利用binder与各个宿主通信,宿主只需嵌入60KB的库即可拥有完整的直播能力,支持多账号管理,多宿主同时使用。项目代码严格模块化,有较好的维护性和扩展性。

 		负责小米直播的插件化能力开发,在两边框架融合成本很高的前提下,以插件化的方式极小地代价完成YY直播在小米直播的接入。
 \end{itemize}
 \end{itemize}


{\bf 小米互娱,高级软件工程师,北京 \hfill  2015.06 $\sim$ 2018.8}
 \vspace{3pt}

 \begin{itemize}
 \item \makebox[7cm][l]{\textit{新米聊App\hfil}} 2017.08 $\sim$ 2018.02
 	\vspace{-3pt}
 	\begin{itemize}
 		\item 
 		主要负责新米聊核心IM消息系统的设计与实现,负责单聊、群聊、群管理等业务,以及会话内各种消息类型的支持。设计了消息补洞,推拉结合等一系列策略保证消息的到达率,对消息到达率和会话交互体验负责。

 \end{itemize}
 \end{itemize}

 \begin{itemize}
 \item \makebox[7cm][l]{\textit{小米直播App\hfil}} 2016.02 $\sim$ 2017.08
 	\vspace{-3pt}
 	\begin{itemize}
 		\item 
 		负责直播小视频、消息通道、礼物商城、主页频道、直播间推拉流等模块的架构与开发。
 		
 		在组期间,组织组内成员使用rxjava、mvp对项目进行代码重构,使用router分割解耦业务模块,使得每个模块能够独立运行,并行开发。并为组内沉淀了一些如大图浏览,视频秒开,音视频录制合成等技术组件。

 		%包大小进行优化,使用fresco+subsample优化看大图组件,使用mat、hierarchyview、traceview、block+leakcanary等工具优化app性能。使用预拉取、cache、cdn预加载等手段优化视频秒开体验,基于MediaCodec为组内沉淀了一些音视频技术。

 		%目前带领3人小组开发米聊消息系统到直播的迁移工作。 远程控制项目,自定义涂鸦 贴纸view(判断一个点是否在矩形内)。

 \end{itemize}
 \end{itemize}
 
\begin{itemize}
\item \makebox[7cm][l]{\textit{milink sdk的研发\hfil}} 2015.07 $\sim$ 至今
	\vspace{-3pt}
	\begin{itemize}
		\item 
		一个通用的长连接通讯组件,将收发消息与具体业务剥离,基于tcp自定义协议,为宿主app提供完整的上下行数据能力,本人负责android sdk的开发,研发ip跑马、动态心跳、双进程保活、双通道、匿名通道,异常容灾等功能。发送消息的成功率在99.9\%。功能稳定,处于维护阶段,为集团的很多app提供数据通道能力。
\end{itemize}
\end{itemize}

\section{掌握技能}
精通java与kotlin语言
精通直播系统与IM系统的设计
精通模块化组件化的解耦合设计
精通长链接组件的设计
熟悉android系统原理
熟悉插件化框架
了解flutter跨平台混合开发
了解gradle plugin与intellij Plugin开发
熟悉java反射与注解原理

\section{教育经历}
中国科学技术大学  ~~{\sl 计算机硕士} ~~ 2012.09 $\sim$ 2015.07

安徽大学 ~~ {\sl 计算机本科} ~~2008.09 $\sim$ 2012.07
%\begin{itemize}
%\item \makebox[7cm][l]{\textit{Shank开发~~实习\hfil}} 2014.10 $\sim$ 2015.02
%	\vspace{-3pt}
%	\begin{itemize}
%		\item 
%		一个孵化项目,仿snapchat做一款基于场景I/M聊天的阅后即焚的app。
%\end{itemize}
%\end{itemize}

 
 %{\bf 科大讯飞,平嵌终端部,合肥,实习\hfill 2013.06 $\sim$ 2013.9}
%	\begin{itemize}
%		\item 
%		参与掌上语厅App的开发。负责界面与业务逻辑的开发以及语音模块的集成。
%
%\end{itemize}

%\section{获得奖项}
%2015小米互娱新人奖

%2016小米互娱优秀员工奖


 \section{自我评价}
较强的工程能力,皮实负责,热爱学习,喜欢钻研,自我驱动,不惧棘手难题,持续关注新技术发展。
%希望能到一个技术氛围浓厚的组里,大家一起交流成长。

%\section{PATENTS}
%One patent filed for Intel work
 
\end{resume}    
\end{document}
